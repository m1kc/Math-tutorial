\documentclass[a5paper,10pt,pagesize,DIV=classic]{scrbook}
\usepackage{cmap}
\usepackage[T2A]{fontenc}
\usepackage[pass]{geometry}

\usepackage{graphicx}
\graphicspath{{images/}}

\usepackage[utf8]{inputenc}
\usepackage[english,russian]{babel}
\usepackage{tikz}
\usepackage{calc}

\usepackage[raggedright,small]{titlesec}
\usepackage[dotinlabels]{titletoc}

%\titlelabel{\thetitle.\quad\thispagestyle{fancy}}

\usepackage{ccaption}
\captiondelim{. }

\usepackage[bookmarks=false, a4paper, colorlinks, unicode, pdfstartview=FitH, pdftex]{hyperref}
\hypersetup{
plainpages=true,
linkcolor=blue,
citecolor=red,
menucolor=blue,
pdfnewwindow=true
}

\clubpenalty=10000
\widowpenalty=10000

\lccode`\-=`\-
\lccode`\+=`\+
\defaulthyphenchar=127
\hfuzz=1.5pt

\usepackage{amsmath}
\usepackage{amssymb}
\usepackage{amsthm}
\usepackage{amstext}
\usepackage{hyperref}


%% use this instead of slhape or textit for inline definitions %%
\newcommand{\term}[1]{\textit{#1}}

%% definition for theorems, excercies, etc. %%
\theoremstyle{plain}
\newtheorem{thm}{Теорема}[chapter]

\theoremstyle{definition}
\newtheorem{exercise}{Упражнение}[chapter]
\newtheorem{problem}{Задача}[chapter]
\newtheorem{example}{Пример}[chapter]
\newtheorem{definition}{Определение}[chapter]

%% itemize use dashes instead of bullets%%
\def\labelitemi{---}


\begin{document}

\title{Учебник по математике}
\author{Роман Добровенский\\ \\ \url{heller@heller.ru}\\ \url{http://heller.ru}}
\date{2012--2013, Москва}
\maketitle

\tableofcontents

\input{introduction}

\chapter{Логика}
Первая глава даёт неформальное описание основных законов логики. Главная цель~--- дать какую-то интуицию об излагаемых понятиях и о принципах логических выводов, прежде чем мы всё это формализуем и используем в конце второй главы. Если вам не особо интересны основания математики, то вы можете сразу переходить к середине третьей главы и читать более прикладные вещи, обращаясь сюда по мере необходимости как к справочнику.

Наиболее важными являются первый и четвёртый параграфы. Остальные параграфы имеют факультативный характер и не обязательны к прочтению.

\input{logic-basics}
\input{logic-functions}
\input{logic-puzzle}
\input{logic-predicates}
\section{Теории и выводимость}

Этот и следующий параграфы делают попытку довольно неформально и на пальцах объяснить вещи, которые обычно изучаются лишь на специализированных курсах университетов и очень в формализованном виде. Если вы ничего здесь не поймете, материал возможно станет понятнее после прочтения второй главы.

{\bfseries Определение.} {\slshape Правилом вывода}, или {\slshape трансформации формул,} будем называть правило, которое  множеству формул ставит в соответствие новую формулу.

{\bfseries Определение.} Правило вывода называется {\slshape корректным}, если при условии истинности всех начальных формул, будет истинной и формула, которая следует по правилу вывода.

Это очень важная характеристика правил вывода, поскольку она гарантирует, что из формул, которые при определенных значениях переменных являются истинными, мы всегда будем выводить истинные формулы при тех же значениях переменных.

Записывать правила вывода мы будем как список начальных формул, по одной формуле в каждой строке, и отделенных подчеркиванием от результата применения правила.

Вот вам для начала несколько примеров корректных правил трансформации:

1) Сокращение двойного отрицания:

$\begin{array}{l} \neg\neg\phi\\ \hline \phi\end{array}$

2) Введение двойного отрицания:

$\begin{array}{l}\phi\\ \hline\neg\neg \phi\end{array}$

3) Введение конъюнкции

$\begin{array}{l}\phi \\ \psi\\ \hline \phi\wedge \psi\end{array}$

4) Сокращение конъюнкции

$\begin{array}{l}\phi \wedge \psi\\ \hline \phi\end{array}$

5) Введение дизъюнкции

$\begin{array}{l}\phi \\ \hline \phi \vee \psi\end{array}$

6) Дизъюнктивный силлогизм

$\begin{array}{l}\phi \vee \psi\\ \neg\phi \\ \hline \psi\end{array}$

7) Сокращение эквиваленции:

а) $\begin{array}{l}\phi \leftrightarrow \psi\\ \phi \\ \hline \psi\end{array}$

б) $\begin{array}{l}\phi \leftrightarrow \psi\\ \neg\phi \\ \hline \neg\psi\end{array}$

в) $\begin{array}{l}\phi \leftrightarrow \psi\\ \phi \vee \psi \\ \hline \phi \wedge \psi\end{array}$

г) $\begin{array}{l}\phi \leftrightarrow \psi\\ \neg\phi \vee \neg\psi \\ \hline \neg\phi \wedge \neg\psi\end{array}$

Я приводил только односторонние правила. Например, при сокращении эквиваленции, я указывал только на сокращение с одной стороны. Можно привести правило и для сокращения с другой стороны.

{\bfseries Упражнение.} Докажите, что приведенные правила вывода действительно корректны.<strong>
</strong>

{\bfseries Определение.} {\slshape Теорией} мы будем называть такое множество формул, что при применении к любым формулам этой теории правил трансформации, мы всегда будем получать формулы этой же теории.

{\bfseries Определение.} Теория называется {\slshape аксиоматизируемой}, если возможно выделить конечный набор формул, называемых {\slshape аксиомами}, такой, что все остальные формулы теории возможно получить, применяя последовательность правил вывода к аксиомам.

Эти определения могут показаться странными и непонятными. Интуитивно об этом проще думать так: в начале задается некоторый набор аксиом, а уже из них пользуясь правилами вывода выводятся различные формулы (они так же называются {\slshape теоремами}). Объединение всех аксиом со всеми теоремами, из них выводимыми (это бесконечное множество), как раз и называется аксиоматизируемой теорией (неаксиоматизируемые теории нам в этом курсе не интересны).

Существенно так же заметить, что если теория аксиоматизируема, то сам набор аксиом для нее не определен однозначно — одна и та же теория может аксиоматизироваться разными способами.

Факт принадлежности формулы к теории записывается как $T\vdash\phi$ (это по сути то же самое, что и $\phi \in T$). При этом говорят, что формула $\phi$ {\slshape выводима, }или является{\slshape  синтаксическим следствием,} в теории $T$. Часто так же отдельно оговаривают, каким множеством правил вывода пользуется теория. Логично, что от набора правил вывода состав теории очень сильно зависим. Мы не будем себя ограничивать и будем говорить, что у нас во всех теориях используются все правила вывода, упомянутые в курсе.

Удобно считать, что для любой тавтологии определено правило вывода из пустого набора формул. То есть, например, если взять тавтологию $a\oplus b \leftrightarrow (a\vee b) \wedge (\neg a \vee \neg b)$, то можно написать следующее «правило вывода»:

$\vdash \phi\oplus \psi \leftrightarrow (\phi\vee \psi) \wedge (\neg \phi \vee \neg \psi)$

Отсутствие указания на теорию слева от знака выводимости говорит о том, что данная формула содержится в любой теории. (Таким образом возможно построение аксиоматизируемой теории с пустым набором аксиом — такая теория будет содержать все логические тавтологии).

В качестве примера всего понаписанного, давайте докажем, что если $T\vdash \phi\oplus\psi$, то тогда $T,\phi\vdash\neg\psi$ (запись слева от знака выводимости говорит о том, что мы рассматриваем теорию $T$, дополненную формулой $\phi$). Причем докажем это не по таблице истинности, а именно с точки зрения синтаксической выводимости.

1) $T \vdash \phi \oplus \psi$ — это наше условие.

2) $T \vdash \phi\oplus \psi \leftrightarrow (\phi\vee \psi) \wedge (\neg \phi \vee \neg \psi)$ - тавтология, которую мы намереваемся использовать.

3) $T\vdash (\phi\vee \psi) \wedge (\neg \phi \vee \neg \psi)$ — исходя из 1) и 2) по правилу о сокращении эквиваленции.

4) $T\vdash \neg \phi \vee \neg \psi$ — исходя из 3) по правилу о сокращении конъюнкции.

5) $T, \phi \vdash \neg \phi \vee \neg \psi$ — если добавить к теории формулу, то все формулы, выводимые ранее, останутся выводимы.

6) $T, \phi \vdash \neg \psi$ —  по правилу о дизъюнктивном силлогизме.

Что и требовалось доказать.

{\bfseries Упражнение.} Попробуйте доказать $T\vdash \phi\oplus\psi$ исходя из двух условий $T, \phi \vdash \neg \psi$ и $T, \psi \vdash \neg \phi$. Возможно ли вывести то же самое, отказавшись от одного из условий?

{\bfseries Упражнение.} {\slshape Прошлое упражнение было удалено отсюда из-за неустранимой ошибки в условии. Скоро вставлю сюда что-нибудь новое.}

В качестве еще одного примера докажем довольно концептуальную штуку. Пусть наша теория такая, что одновременно $T\vdash\phi$ и $T\vdash\neg\phi$ (то есть из одной теории выводится одновременно и некоторая формула $\phi$, и ее отрицание). Тогда можно провести следующую цепочку рассуждений:

1) $T\vdash\phi$ — по условию.

2) $T\vdash\phi\vee\psi$ — введение дизъюнкции. Здесь $\psi$ — произвольная формула.

3) $T\vdash\neg\phi$ — по условию.

4) $T\vdash\psi$ — исходя из 2) и 3) и дизъюнктивного силлогизма.

Таким образом получилось, что из теории $T$ мы можем вывести совершенно произвольную формулу. Такие теории естественно имеют мало смысла, и называются {\slshape противоречивыми}. Характеризуются они выводимостью из них одновременно какой-либо формулы и ее отрицания.

Если теория $T$, с которой мы работаем, непротиворечива, то из сказанного можно заключить, что корректно так же следующее правило вывода:

$\begin{array}{l} T,\phi\vdash\psi\\ T,\phi\vdash\neg\psi\\ \hline T\vdash\neg\phi\end{array}$

Следующие правила вывода потребуют некоторого напряжения:

{\bfseries Universal Instantiation (UI):} $\begin{array}{l}\forall x\phi(x)\\ \hline\phi(\alpha)\end{array}$

{\bfseries Existential Instantiation (EI):} $\begin{array}{l}\exists x\phi(x)\\ \hline\phi(\alpha)\end{array}$

{\bfseries Existantial Generalization (EG):} $\begin{array}{l}\phi(\alpha)\\ \hline\exists x\phi(x)\end{array}$

{\bfseries Universal Generalization (UG)}: $\begin{array}{l}\phi(\alpha)\\ \hline\forall x\phi(x)\end{array}$

Эти правила работают далеко не всегда. Чтобы их применять, необходимо быть уверенным в том, что имена переменных, которые вы используете в ваших манипуляциях с кванторами и переименованиями, нигде в теории больше не задействованы, так что ваши манипуляции не повлияют на остальную теорию. Я для начала покажу пример вывода с использованием этих правил, а затем мы немного детальнее обсудим откуда они берутся и как интерпретируются.

Вот как можно решить упражнение из прошлого параграфа:

1) $T\vdash\forall x, (P(x)\wedge Q(x))$ — начальное условие.

2) $T\vdash P(a)\wedge Q(a)$ — применение UI

3) $T\vdash P(a)$ — сокращение конъюнкции

4) $T\vdash\forall x, P(x)$ — применение UG

5) $T\vdash Q(a)$ — еще раз сокращение конъюнкции для 2), но уже с другой стороны.

6) $T\vdash \forall x, Q(x)$ — применение UG

7) $T\vdash (\forall x, P(x)) \wedge (\forall x, Q(x))$ — введение конъюнкции для 4) и 6)

Это доказательство в одну сторону. В другую стороны оно выполняется точно так же, но в обратном порядке и с другими названиями для правил. Собственно что и требовалось доказать.

{\bfseries Упражнение.} Докажите с помощью правил вывода для кванторов остальные формулы упражнений прошлого параграфа.

Теперь давайте рассмотрим пример, когда применять эти правила нельзя:

1) $T\vdash \exists x\exists y, x \not= y$ — пусть это наше предположение

2) $T\vdash a \not= b$ — результат двойного применения EI

3) $T\vdash \forall b, b \not= b$ — результат применения UG к $a$.

Последний вывод явно не верен, впрочем, и ошибка наша очевидна: мы переименовали $a$ в переменную, которая уже задействована, чем внесли разлад в теорию. Однако если выбрать другое имя, то получится не лучше:

3) $T\vdash \forall z, z \not= b$ — вторая попытка с другим именем

Тоже ничего хорошего — наше изначальное предположение было в том, что просто существуют элементы, не равные друг другу, а не то что все они не равны одному конкретному элементу (последнее вообще делает нашу теорию противоречивой). В данном случае мы опять же не имели права применять операцию UG к $a$ по той причине, что эта переменная не совсем независимая — изначально при переименовании мы подразумевали, что новое введенное имя $a$ — это имя некоторой переменной, которая не равна $b$. И это предположение об $a$ никуда не делось — следовательно мы не имели права применять UG.

Первый пример отличается от этого тем, что там переменная $a$ изначально имела квантор всеобщности, и применяя операцию UI мы не накладывали на нее никаких ограничений, а следовательно могли его затем обратно смело возвращать. На практике в основном UI и UG так и используются — вначале убирается квантор всеобщности, затем над формулой проводятся какие-то манипуляции (с квантором мы бы их не могли сделать), и в конце квантор возвращается.

Второй способ применить UG не рискуя внести противоречия в теорию — это применить его к формуле, которая изначально появилась в теории из тавтологии. Вводя тавтологии мы можем задавать произвольные имена переменным, и они ни от чего не будут зависеть.

Почему эти правила вообще работают? UI, EI, EG, думаю, особых сомнений не вызывают. Например, UI применялся еще философами Древней Греции, откуда нам досталось классическое рассуждение: «Все люди смертны. Сократ — человек. Следовательно, Сократ смертен». Если записать это на языке логики, то это в точности пример применения правила UI. Для остальных правил так же можно придумать подобную интуицию.

Сомнения может вызывать разве что правило UG. На первый взгляд оно кажется очень странным: из того, что $P(a)$ верно для одного какого-то конкретного $a$, оно оказывается верным сразу для всех $a$. «Из того, что я дурак, следует, что и все дураки» — явно не верное рассуждение. Однако надо помнить, что на применение UG накладывается жесткое условие, чтобы сам объект утверждения, к которому мы применяем правило, был независим от остальных теорем. А раз в независимости от остальных теорем у нас оказался выводим $P(a)$, то и для любого другого объекта мы тоже могли это вывести. На практике это либо возврат ранее снятого квантора, либо введение тавтологии.

{\bfseries Упражнение.} Развлеките сами себя. Придумайте какую-нибудь содержательную теорию.

\section{Семантика}

{\bfseries Определение.} {\slshape Моделью} называется множество высказываний.

Если в модели $M$ истинно высказывание $s$, то это обозначается как $M\models s$.

Если любая формула теории $T$ истинна в модели $M$, то говорят, что $M$ является моделью $T$.

У каждой теории может быть много моделей, которые могут иметь как вообще различный набор истинных высказываний, так и просто разную трактовку. Рассмотрим простенькую теорию, для которой $T\vdash \neg a \vee (b \wedge c)$ и $T\vdash a \oplus d$.

Можно рассмотреть модель этой теории, в которой $a$ = «светит солнце», $b$ = «девки по улице гуляют», $c$ = «птички поют», $d$ = «Вася по улице гуляет». Формулы теории $T$ соответственно обозначают, что если на лице светит солнце, то девки гуляют и птички поют, и что у Васи аллергия на Солнце, и пока все гуляют, он дома сидит и книжку по математике читает. При этом в нашей модели $M_0\models a, b, c, \neg d$. Это один из вариантов, но важно, что для каждого высказывания мы точно можем определить его истинность в данной модели.

Или можно рассмотреть модель, в которой $a$ = «растет курс рубля», $b$ = «повышается импорт», $c$ = «снижается спрос на отечественную продукцию», $d$ = «растет производство внутри страны». Теоремы на этот раз можно интерпретировать как то, что из роста курса рубля повышается импорт и снижается спрос на отечественную продукцию, а так же то что рост курса рубля и рост производства внутри страны не могут происходить одновременно. Это очень утрированная теория с точки зрения экономики, но для учебника математики пойдет. Можно считать, что в нашей конкретной модели $M_1\models \neg a, b, \neg c, d$.

Как видно, модели одной и той же теории могут различаться не только физической интерпретацией, но и истинностью отдельных высказываний. В примере выше у нас в одной модели было $M_0\models a$, а в другой $M_1\models \neg a$, и обе эти модели удовлетворяли теории $T$. Отсюда, например, можно сразу заключить, что высказывание $a$ не выводимо само по себе в теории $T$, что обозначается как $T\not\vdash a$.

Множество всех моделей теории $T$ обозначается как $\mathrm{Mod}(T)$. Например, для рассмотренной выше теории,\\
$\mathrm{Mod}(T) = \{\{a, b, c, \neg d\}, \{\neg a, b, c, d\}, \{\neg a, \neg b, c, d\}, \{\neg a, b, \neg c, d\},\\ \{\neg a, \neg b,\neg c, d\}\}$, плюс у каждой модели могут быть разные интерпретации.

{\bfseries Упражнение.} Докажите, что для противоречивой теории не существует модели.

Возможна и обратная задача. Если задано множество моделей, то по ним можно построить теорию, которая в точности будет удовлетворять этим моделями и никаким другим. Если $S$ — множество моделей, то теория, по ним построенная обозначается как $\mathrm{Th}(S)$.

{\bfseries Упражнение.} Пусть $S = \{\{a, \neg b, c\}, \{a, \neg b, \neg c\}, \{\neg a, b, \neg c\}\}$. Аксиоматизируйте $\mathrm{Th}(S)$ несколькими разными способами.

{\bfseries Теорема.} Если $T_1 \subset T_0$, то $\mathrm{Mod}(T_0) \subset \mathrm{Mod}(T_1)$.

{\bfseries Доказательство.} Каждую формулу теории можно считать предикатом, который задает подмножество на множестве моделей. Запись $T_1 \subset T_0$ можно трактовать как то, что $T_0$ получается из $T_1$ добавлением предикатов, которые задают подмножество в множестве $\mathrm{Mod}(T_1)$. \qed

Добавляя к теориям новые формулы и наблюдая, как это влияет на множество моделей самой теории, можно делать выводы о самой теории. Например, если $\mathrm{Mod}(T, \phi) = \mathrm{Mod}(T, \psi)$, то есть при добавлении к теории формул $\phi$ и $\psi$ мы получаем одни и те же множества моделей, то можно говорить в некотором смысле об эквивалентности формул $\phi$ и $\psi$ — в любой модели теории они будут либо одновременно истинны, либо одновременно ложны.

{\bfseries Теорема.} Если $T\vdash \phi \leftrightarrow \psi$, то $\mathrm{Mod}(T, \phi) = \mathrm{Mod}(T, \psi)$.

{\bfseries Доказательство.} По правилу сокращения эквиваленции $T, \phi\vdash \psi$ и $T, \psi \vdash \phi$, то есть при добавлении к теории $T$ формул $\phi$ либо $\psi$, мы все равно в результате получаем одни и те же множества формул нашей теории. \qed

А вот обратное вообще говоря не всегда верно. Если две формулы одновременно либо ложны либо истинны (множества моделей для них совпадают), то еще совершенно не факт, что это возможно вывести в нашей теории по правилам вывода. По этой причине различают два понятия — {\slshape синтаксис}, имеющий дело с выводимостью, и {\slshape семантику}, имеющую дело с моделями. В случае эквивалентности, синтаксическая эквивалетность утверждает, что формулы могут быть выведены одна из другой и наоборот, а семантическая эквивалентность утверждает, что они всегда либо одновременно ложны, либо одновременно истинны.

По аналогии говорят, что $\psi$ является {\slshape семантическим следствием} $\phi$, если всегда, когда истинно $\phi$, истинно так же и $\psi$, что в терминах моделей записывается как $\mathrm{Mod}(T, \phi) \subset \mathrm{Mod}(T, \psi)$.

Опять же, если $\psi$ выводимо из $\phi$ в теории $T$, то оно же автоматически является и синтаксическим следствием $\phi$. Обратное не всегда верно.

{\bfseries Определение.} Теория называется {\slshape полной}, если из семантического следствия вытекает выводимость.

Пример неполной теории мы пока не сумеем на самом деле привести. {\slshape Теорема Гёделя о полноте} утверждает, что любая теория в рамкой той логики, которую мы до сих пор видели, является полной. Теорему эту мы доказывать не будем (вы можете попробовать это самостоятельно), так как в контексте нашего курса она не особо интересна — все последующее изложение у нас будет вестись в контексте неполной теории.

Вернемся к нашим следствиям. Мы сформулировали понятие семантической эквивалентности для моделей, и оказалось, что это тесно связано с обычной операцией эквиваленции. Введем теперь по аналогии с эквиваленцией логическую операцию для следствия, которая бы находилась в соответствии с семантикой нашей теории. Научно ее называют «импликацией», от латинского «imlicatio», что значит «следствие».

{\bfseries Определение.} {\slshape Импликацией} из $\phi$ в $\psi$ (обозначение $\phi\rightarrow \psi$) мы будем называть такую логическую операцию, для которой истинность $\phi\rightarrow\psi$ эквивалентна семантическому следствию $\mathrm{Mod}(T, \phi) \subset \mathrm{Mod}(T, \psi)$.

То есть можно написать так: $\phi\rightarrow\psi = \mathrm{Mod}(T, \phi) \subset \mathrm{Mod}(T, \psi)$

По этому определению для импликации легко построить таблицу истинности. Пусть, например, у нас задана модель $M\models\phi, \neg\psi$. Очевидно, что при этом свойство $\mathrm{Mod}(T, \phi) \subset \mathrm{Mod}(T, \psi)$ не выполняется, и $\phi\rightarrow\psi = 0$. Аналогично можно рассмотреть другие случаи, и получить следующую таблицу истинности:

$\begin{array}{cc|c}\phi&\psi&\phi\rightarrow\psi \\ \hline 0&0&1 \\ 0&1&1 \\ 1&0&0 \\ 1&1&1\end{array}$

Из всего сказанного и таблицы истинности можно тут же выразить следующие очевидные и логичные свойства:

1) $a \rightarrow b = b \vee \neg a$

2) $\neg(a \rightarrow b) = a \wedge \neg b$

3) $a \rightarrow a$

4) $a \leftrightarrow b = (a \rightarrow b) \wedge (b\rightarrow a)$

5) {\slshape Транзитивность:} $((a \rightarrow b) \wedge (b \rightarrow c)) \rightarrow (a \rightarrow c)$

6) $(a \vee b) \wedge (\neg a \vee c) \rightarrow b \vee c$

7) $(a \rightarrow b \wedge c) \rightarrow (a \rightarrow b)$

8) $a \rightarrow b = \neg b \rightarrow \neg a$

А так же правила вывода:

1) Теорема дедукции:

а) $\begin{array}{l}T, \phi\vdash\psi\\ \hline T\vdash\phi\rightarrow\psi\end{array}$

б) $\begin{array}{l}T\vdash\phi\rightarrow\psi\\ \hline T, \phi\vdash\psi\end{array}$

2) Modus ponens:

$\begin{array}{l}\phi\\ \phi\rightarrow\psi\\ \hline \psi\end{array}$

3) Modus tollens:

$\begin{array}{l}\neg\psi\\ \phi\rightarrow\psi\\ \hline \neg\phi\end{array}$

4) Анализ частных:

$\begin{array}{l}\phi\vee\psi\\ \phi\rightarrow\chi\\ \psi\rightarrow\chi\\ \hline\chi\end{array}$

5) Введение эквиваленции:

$\begin{array}{l}\phi\rightarrow\psi\\\psi\rightarrow\phi\\ \hline\phi\leftrightarrow\psi\end{array}$

Ну и свойства, связанные с кванторами:

1) $\forall x, (P\rightarrow Q(x)) = P\rightarrow (\forall x, Q(x))$

2) $\forall x, (P(x)\rightarrow Q) = (\exists x, P(x))\rightarrow Q$

Докажем для примера последнее свойство:

1) $T\vdash\forall x, (P(x)\rightarrow Q)$ — начальное условие

2) $T\vdash P(a)\rightarrow Q$ — правило UI

3) $T, (\exists x, P(x))\vdash P(a)$ — правило EI, примененное к $\exists x, P(x)$

4) $T, (\exists x, P(x)) \vdash Q$ — правило modus ponens для 2) и 3)

5) $T\vdash(\exists x, P(x))\rightarrow Q$ — дедукция

Что и требовалось. И теперь доказательство в другую сторону:

1) $T\vdash(\exists x, P(x))\rightarrow Q$ — начальное условие

2) $T, P(a) \vdash \exists x, P(x)$ — правило EG для P(a)

3) $T, P(a) \vdash Q$ — modus ponens для 1) и 2)

4) $T\vdash P(a)\rightarrow Q$ — дедукция

5) $T\vdash \forall x, (P(x)\rightarrow Q)$ — правило UG

Можно было, впрочем, решить задачу и в лоб:

$\forall x, (P(x)\rightarrow Q) = \forall x, (\neg P(x) \vee Q) = (\forall x, \neg P(x)) \vee Q = (\neg\exists x, P(x))\vee Q = (\exists x, P(x))\rightarrow Q$

{\bfseries Упражнение.} Докажите все остальные правила для импликации, приведенные выше.

{\bfseries Упражнение.} Выясните, в каких случаях неравенства в последнем упражнении §1.4, можно заменить на импликацию.

{\bfseries Упражнение.} Пусть $T\vdash\phi\rightarrow\psi$. Мы каким-то образом работаем с теорией $T$ в предположении $\phi$. Вдруг мы обратили внимание на нашу импликацию, и решили, что работать с предположением $\psi$ вместо $\phi$ было бы удобнее. Мы так и поступаем: забываем о предположении истинности $\phi$ и считаем истинным $\psi$. Покажите, что множество выводимых формул при этом возможно (но не обязательно), сократится.

Последнее упражнение — это пример ошибочных рассуждений и нравственное наставление: так как описано в упражнении никогда нельзя поступать, это в крайней степени порочно и греховно. Чтобы ничего не потерять в теории, мы не должны выкидывать никаких формул и предположений, лишь добавлять гипотезы к уже имеющемуся у нас набору по необходимости, либо использовать теорему о дедукции и получать дополнительные правила импликации.

\input{logic-paradoxes}
\input{logic-applications}


\chapter{Множества}
В этой главе очень формально и кратко будут изложены основные понятия множеств. Запомнить и понять сходу всё, что здесь написано, очень сложно, поэтому лучшей стратегией будет поверхностно ознакомиться с содержимым главы, а затем возвращаться к ней по мере необходимости как к справочному материалу. Важными параграфами являются 1, 2 и 4. Остальные можно пропустить, если будут возникать трудности при прочтении.

\input{sets-basics}
\input{sets-relations}
\input{sets-graphs}
\input{sets-functions}
\input{sets-formalism}
\input{sets-incompleteness}
\input{sets-history}


\chapter{Натуральные числа}
В этой части наконец-то вводится понятие натурального числа. Основные две темы ~--- само понятие натурального числа (включая минимум из теории чисел) и комбинаторика. Базовые комбинаторные формулы сами по себе часто оказываются полезны в самых разных областях математики, так же на них легко и приятно отрабатывать основные навыки работы с базовыми математическими объектами. В более абстрактные области мы уйдем со следующей главы, эта же глава будет во многом простая, расслабляющая и легкая для чтения (возможно, кроме первого параграфа, который в принципе многим читателям будет необязателен).

\input{naturals-definitions}


\end{document}
